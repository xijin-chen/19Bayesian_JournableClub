\usepackage{xifthen}            % for \isempty
\usepackage{array}
\usepackage{amssymb}

%%%%%%%%%%%%%%%%%%%%%%%%%%%%% Kommentare %%%%%%%%%%%%%%%%%%%%%%%%%%%%%

% Zum Kommentieren
\newcommand{\komm}[1]{%
  \marginpar{\fbox{% mit Rahmen
    \begin{minipage}{1.4cm}  % Für automatischen Zeilenumbruch im Kommentar
      {\footnotesize\bfseries #1}
    \end{minipage}
  }
 }
} 


% Zum Auskommentieren
\newcommand{\blanco}[1]{  } 
%\usepackage[sumlimits, intlimits, namelimits]{amsmath} 

%%%%%%%%%%%%%%%%%%%%%%%%%%% Formatierung %%%%%%%%%%%%%%%%%%%%%%%%%%%%%%

%\newcommand{\alert}{}
\newcommand{\gqs}{}

% für englische Begriffe: Achtung, falls sich ein Wort anschließt so:
% \english{engWord}{}, d.h. Klammern nicht vergessen  
\newcommand{\english}[1]{\glqq #1\grqq}

% für lateinische Begriffe, z. B. a posteriori, ad hoc, etc
\newcommand{\latin}[1]{\textit{#1}}

% für Definitionen
\newcommand{\define}[1]{\emph{#1}} % jetzt ohne index, das muss manuell gemacht werden

% angeben von Personennamen
\newcommand{\name}[1]{\textsc{#1}} 

% Abkürzungen richtig setzen ("z.B.") 
\newcommand{\abks}[1]{\mbox{\scriptsize #1}\xdot}
\newcommand{\abk}[1]{\mbox{#1}\xdot}
\DeclareRobustCommand\xdot{\futurelet\token\Xdot}
\def\Xdot{%
  \ifx\token\bgroup.%
  \else\ifx\token\egroup.%
  \else\ifx\token\/.%
  \else\ifx\token\ .%
  \else\ifx\token!.%
  \else\ifx\token,.%
  \else\ifx\token:.%
  \else\ifx\token;.%
  \else\ifx\token?.%
  \else\ifx\token/.%
  \else\ifx\token'.%
  \else\ifx\token).%
  \else\ifx\token-.%
  \else\ifx\token+.%
  \else\ifx\token~.%
  \else\ifx\token.%
  \else.\ %
  \fi\fi\fi\fi\fi\fi\fi\fi\fi\fi\fi\fi\fi\fi\fi\fi%
}

\newcommand{\etc}{\abk{etc}}
\newcommand{\Prof}{\abk{Prof}}
\newcommand{\Dr}{\abk{Dr}}
\newcommand{\vgl}{\abk{vgl}}
\newcommand{\zB}{\abk{z.\,B}}
\newcommand{\bzgl}{\abk{bzgl}}
\newcommand{\bzw}{\abk{bzw}}
\newcommand{\dH}{\abk{d.\,h}}
\newcommand{\ua}{\abk{u.\,a}}
\newcommand{\ca}{\abk{ca}}
\newcommand{\ggf}{\abk{ggf}}
\newcommand{\eg}{\abk{\latin{e.\,g}}}
\newcommand{\ie}{\abk{\latin{i.\,e}}}
\newcommand{\cf}{\abk{\latin{cf}}}

% Grafikskalierung
\newcommand{\graphicSize}[1]{\setkeys{Gin}{width = #1\textwidth, keepaspectratio}}
\newcommand{\fullwidth}{\setkeys{Gin}{width = \textwidth, keepaspectratio}}

\newlength{\halbebreite}
\setlength{\halbebreite}{\textwidth / 2 - 0.5cm}
\newcommand{\halfwidth}{\setkeys{Gin}{width = \halbebreite, keepaspectratio}}

% Referenzierung von Listenelementen
%\newcommand{\subref}[1]{\ref{#1})}

%%%%%%%%%%%%%%%%%%%%%%%%%%%%%% Befehle %%%%%%%%%%%%%%%%%%%%%%%%%%%%%%%

% Verteilungen (- bedeutet: im Appendix eingetragen)
\DeclareMathOperator{\Ber}{B} % Bernoulli - 
\DeclareMathOperator{\Bin}{Bin} % Binomial Distribution - 
\DeclareMathOperator{\Cauchy}{C} % Cauchy Distribution (special Student -
                                % dist.) 
\DeclareMathOperator{\Par}{Par} % Pareto Distribution
\DeclareMathOperator{\Mult}{M} % Multinomialverteilung - 
\DeclareMathOperator{\NegBin}{NBin} % Negative Binomial - 
\DeclareMathOperator{\HypGeom}{HypGeom} % Hypergeometric Distribution - 
\DeclareMathOperator{\NCHypGeom}{NCHypGeom} % Noncentral hypergeometric Distribution - 
\DeclareMathOperator{\Geom}{Geom} % Geometric Distribution - 
\DeclareMathOperator{\Po}{Po} % Poisson Distribution - 
\DeclareMathOperator{\Exp}{Exp} % Exponential Distribution -
\DeclareMathOperator{\Nor}{N} % Normal -
\DeclareMathOperator{\LN}{LN} % Log-Normal - 
\DeclareMathOperator{\HN}{HN} % Halb-Normal - 
\DeclareMathOperator{\FN}{FN} % gefaltet Normal - 
\DeclareMathOperator{\Gumbel}{Gu} % Gumbel - 
\DeclareMathOperator{\F}{F} % F - 
\DeclareMathOperator{\stud}{t} % Student - 
\DeclareMathOperator{\Stud}{\stud}  
\DeclareMathOperator{\Log}{Log} % Logistische Verteilung - 
\DeclareMathOperator{\Uni}{U} % Uniform -
\DeclareMathOperator{\Ga}{G} % Gamma - 
\DeclareMathOperator{\IG}{IG} % Invers-Gamma - 
\DeclareMathOperator{\Gg}{Gg} % Gamma-Gamma - 
\DeclareMathOperator{\Be}{Be} % Beta - 
\DeclareMathOperator{\BeB}{BeB} % Beta-Binomial - 
\DeclareMathOperator{\PoG}{PoG} % Poisson-Gamma - 
\DeclareMathOperator{\Wb}{Wb} % Weibull - 
\DeclareMathOperator{\Dir}{D} % Dirichlet - 
\DeclareMathOperator{\Wish}{Wi} % Wishart 
\DeclareMathOperator{\InvWish}{IWi} % Inverse Wishart
\DeclareMathOperator{\MultDir}{MD} % Multinomial-Dirichlet -
\DeclareMathOperator{\NoG}{NG} % Normal-Gamma - 

% Operatoren
\DeclareMathOperator{\Var}{Var} % Varianz
\DeclareMathOperator{\E}{\mathsf{E}} % Erwartungswert
\newcommand{\KLD}[2]{\mathsf{D}(#1 \parallel{} #2)} % new KL discrepancy
\DeclareMathOperator{\Cov}{Cov} % Covariance
\DeclareMathOperator{\Corr}{Corr} % Correlation 
\DeclareMathOperator{\se}{se}   % standard error
\DeclareMathOperator{\sign}{sign} % signum
\DeclareMathOperator{\logit}{logit} % logit-Funktion
\DeclareMathOperator{\Mod}{Mod} % Modus
\DeclareMathOperator{\Med}{Med} % Median
\DeclareMathOperator{\diag}{diag} % Diagonalmatrix
\DeclareMathOperator{\trace}{tr} % Spur
\renewcommand{\P}{\operatorname{\mathsf{Pr}}} % Wahrscheinlichkeitsmaß
%\newcommand{\p}{\operatorname{\mathsf{p}}} % Density function
\newcommand{\p}{f}%{\operatorname{{p}}} % Density function
\newcommand{\B}{\operatorname{{B}}} % Beta function
%\newcommand{\Lik}{\operatorname{\mathsf{L}}} % Probability/Density function
\newcommand{\Lik}{L} % Probability/Density function
\DeclareMathOperator{\dotcup}{\dot{\cup}} % disjunkte Vereinigung
\DeclareMathOperator{\arctanh}{arctanh} % arcus tangens hyperbolicus
\DeclareMathOperator*{\argmax}{arg\,max} % argument which maximises
\DeclareMathOperator*{\argmin}{arg\,min} % argument which minimises

\DeclareMathOperator{\BS}{BS} % Brier Score
\DeclareMathOperator{\AS}{AS} % Absolute Score
\DeclareMathOperator{\CRPS}{CRPS} % CRPS
\DeclareMathOperator{\LS}{LS} % Logarithmic Score
\DeclareMathOperator{\SPE}{SPE} % Squared prediction error
\DeclareMathOperator{\SC}{SC} % Sander's calibration
\DeclareMathOperator{\MR}{MR} % Murphy resolution
\DeclareMathOperator{\AUC}{AUC} % Area under the curve
\DeclareMathOperator{\BF}{BF} % Bayes factor
\DeclareMathOperator{\mBF}{mBF} % minimum Bayes factor



% Zum Angeben von Funktionen
\newcommand{\funktion}[5]{%
	\begin{tabular}[t]{lrcl}
	$#1$~: & $#2$ & $\longrightarrow$ & $#3$\\
	& $#4$ & $\longmapsto$ & $#5$
	\end{tabular}}

% für Pfeil mit Erklärung unter einem Formelteil
\newcommand{\underarrow}[2]{%
 \underset{\begin{subarray}{c} \uparrow\\ #1 \end{subarray}}{#2}%
}

% Text über =
\newcommand{\overequal}[1]{\overset{\text{#1}}{=}}

% partielle Abl. von #2 nach #3 mit optionalem Parameter #1 für wievielte Ableitung (default 1)
\newcommand{\partialv}[3][1]{%
% \ifthenelse{#1 = 1}{\frac{\partial\,#2}{\partial\,#3}}{\frac{\partial^{#1} #2}{\partial\,#3^{#1}}}
\ifthenelse{#1 = 1}{\frac{\partial #2}{\partial #3}}{\frac{\partial^{#1} #2}{\partial #3^{#1}}}
} 

% Abl. von #2 nach Skalar #3 mit optionalem Parameter #1 für wievielte Ableitung (default 1)
\newcommand{\partials}[3][1]{%
%% \ifthenelse{#1 = 1}{\frac{d\,#2}{d\,#3}}{\frac{d^{#1} #2}{d\,#3^{#1}}}
\ifthenelse{#1 = 1}{\frac{d #2}{d #3}}{\frac{d^{#1} #2}{d #3^{#1}}}
} 

% partielle Abl. mit separatem Bruch für "nach einem Skalar" mit optionalem Parameter #1 für
% wievielte Ableitung (default 1) 
\newcommand{\dseps}[2][1]{%
% \ifthenelse{#1 = 1}{\frac{d}{d\,#2}}{\frac{d^{#1}}{d\,#2^{#1}}}
\ifthenelse{#1 = 1}{\frac{d}{d #2}}{\frac{d^{#1}}{d #2^{#1}}}
}

% partielle Abl. mit separatem Bruch mit optionalem Parameter #1 für
% wievielte Ableitung (default 1) 
\newcommand{\dsepv}[2][1]{%
% \ifthenelse{#1 = 1}{\frac{\partial\,}{\partial\,#2}}{\frac{\partial^{#1}}{\partial\,#2^{#1}}}
\ifthenelse{#1 = 1}{\frac{\partial}{\partial #2}}{\frac{\partial^{#1}}{\partial #2^{#1}}}
}

%%%%%%%%%%%%%%%%%%%%%%%%%%%%%% Abkürzungen %%%%%%%%%%%%%%%%%%%%%%%%%%%

% Mengen
\newcommand{\mcf}{\mathcal{F}}
\newcommand{\ve}{\varepsilon}
\newcommand{\C}{\mathbb{C}}
\newcommand{\R}{\mathbb{R}}
\newcommand{\Q}{\mathbb{Q}}
\newcommand{\Z}{\mathbb{Z}}
\newcommand{\N}{\mathbb{N}}
\newcommand{\0}{\emptyset}
\newcommand{\Tau}{\mathcal{T}}

% Quer-Versionen
\newcommand{\deltaq}{\bar{\delta}}
\newcommand{\xq}{\bar{x}}
\newcommand{\Xq}{\bar{X}}
\newcommand{\yq}{\bar{y}}
\newcommand{\Yq}{\bar{Y}}
\newcommand{\eq}{\bar{e}}

% Dach-Versionen
\newcommand{\xd}{\hat{x}}
\newcommand{\Xd}{\hat{X}}
\newcommand{\yd}{\hat{y}}
\newcommand{\Yd}{\hat{Y}}
\newcommand{\bd}{{\hat{\beta}}}
\newcommand{\ad}{{\hat{\alpha}}}
\newcommand{\pid}{\hat{\pi}}
\newcommand{\sd}{{\hat{\sigma}}}
\newcommand{\sda}{\hat{\sigma}_{\hat{\alpha}}}
\newcommand{\sdb}{\hat{\sigma}_{\hat{\beta}}}

\newcommand{\ml}[2][1]{% % für Maximum-Likelihood-Schätzer von #1
\ifthenelse{#1 = 1}%
 {\hat{#2}_{\scriptscriptstyle{\mathrm{ML}}}}% 
 {\hat{#2}^{#1}_{\scriptscriptstyle{\mathrm{ML}}}}% z.B. für sigmadach^2
}
\newcommand{\map}[2][0]{% % für MAP-Schätzer von #1
\ifthenelse{#1 = 0}%
 {\hat{#2}_{\scriptscriptstyle{\mathrm{MAP}}}}% 
 {\hat{#2}_{{\scriptscriptstyle{\mathrm{MAP}}_{#1}}}}% z.B. für sigmadach^2
}
\newcommand{\mpm}[2][0]{% % für MPM-Schätzer von #1
\ifthenelse{#1 = 0}%
 {\hat{#2}_{\scriptscriptstyle{\mathrm{MPM}}}}% 
 {\hat{#2}_{{\scriptscriptstyle{\mathrm{MPM}}_{#1}}}}% z.B. für sigmadach^2
}

%% \newcommand{\ml}[2][1]{% % für Maximum-Likelihood-Schätzer von #1
%% \ifthenelse{#1 = 1}%
%%  {\hat{#2}_{\scriptscriptstyle{\text{ML}}}}% 
%%  {\hat{#2}^{#1}_{\scriptscriptstyle{\text{ML}}}}% z.B. für sigmadach^2
%% }
%% \newcommand{\map}[2][0]{% % für MAP-Schätzer von #1
%% \ifthenelse{#1 = 0}%
%%  {\hat{#2}_{\scriptscriptstyle{\text{MAP}}}}% 
%%  {\hat{#2}_{{\scriptscriptstyle{\text{MAP}}_{#1}}}}% z.B. für sigmadach^2
%% }
%% \newcommand{\mpm}[2][0]{% % für MAP-Schätzer von #1
%% \ifthenelse{#1 = 0}%
%%  {\hat{#2}_{\scriptscriptstyle{\text{MPM}}}}% 
%%  {\hat{#2}_{{\scriptscriptstyle{\text{MPM}}_{#1}}}}% z.B. für sigmadach^2
%% }

\newcommand{\myround}[2][1]{\format{#2,nsmall=#1,digits=#1}}

% Verteilt wie
\newcommand{\simah}{\stackrel{a}{\underset{H_0}{\thicksim}}} % approx. Vtlg. unter H0
\newcommand{\sima}{\mathrel{\overset{\text{a}}{\thicksim}}} % approx. Vtlg.
\newcommand{\simh}{\mathrel{\underset{H_0}{\thicksim}}} % Vtlg. unter H0
\newcommand{\simiid}{\mathrel{\overset{\text{iid}}{\thicksim}}} % iid-verteilt 
\newcommand{\simid}{\mathrel{\overset{\text{id}}{\thicksim}}} % id-verteilt 
\newcommand{\simind}{\mathrel{\overset{\text{ind}}{\thicksim}}} % unabhängig verteilt 
\newcommand{\simcid}{\mathrel{\overset{\text{cid}}{\thicksim}}} % cid-verteilt 
\newcommand{\yobs}{y_{\text{o}}}
\newcommand{\yobstilde}{\tilde{y}_{\text{o}}}
\newcommand{\Yobs}{Y_{\text{o}}}

% Operationen
\newcommand{\given}{\,\vert\,} % für "X gegeben Y" also $X\given Y$ schreiben
\newcommand{\semicolon}{\,;\,} % für "X gegeben Y" also $X\given Y$ schreiben
\newcommand{\s}{\setminus}
\newcommand{\entspricht}{\mathrel{\widehat{=}}}
\newcommand{\abs}[1]{\left\lvert#1\right\rvert} % Absolutbetrag
\newcommand{\absmall}[1]{\lvert#1\rvert} % Absolutbetrag ohne Größenanpassung
\newcommand{\norm}[1]{\left\lVert#1\right\rVert} % Norm
\newcommand{\ceil}[1]{\left\lceil#1\right\rceil} % Ceiling
\newcommand{\floor}[1]{\left\lfloor#1\right\rfloor} % Floor
\newcommand{\sprod}[1]{\left\langle#1\right\rangle} % Skalarprodukt
\newcommand{\sdiff}{\bigtriangleup} % symm. Differenz

% Grenzen 
\newcommand{\cupg}{\bigcup\limits}
\newcommand{\capg}{\bigcap\limits}

% sonstiges
\newcommand{\com}[1]{\left(#1\right)^c} % Complement von #1
\newcommand{\upvp}{\upvarphi}
\newcommand{\vpnorm}{\upvp}
\newcommand{\vp}{\phi}
\newcommand{\vt}{\vartheta}
\newcommand{\midt}[1]{\quad\text{#1}\quad} % spart Schreibarbeit
\newcommand{\glqm}{\text{``}} % für Anführungszeichen im Mathemodus
\newcommand{\grqm}{\text{''}}
\newcommand{\Ind}[2]{\mathsf{I}_{#2}(#1)} % Indikatorfunktion
\newcommand{\IdMat}{\boldsymbol{\mathrm{I}}} % identity matrix

% for diagnostic testing examples
\newcommand{\Dp}{\mbox{$D+$}}
\newcommand{\Dm}{\mbox{$D-$}}
\newcommand{\Tp}{\mbox{$T+$}}
\newcommand{\Tm}{\mbox{$T-$}}

% Local Variables: 
% mode: latex
% TeX-master: "MSI"
% ispell-local-dictionary: "english"
% End: 

%%%%%%%%%%%%%%%%%%%%%%%%%%%%%%%% Ende %%%%%%%%%%%%%%%%%%%%%%%%%%%%%%%%
